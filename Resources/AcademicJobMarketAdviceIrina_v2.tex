\documentclass{article}
\usepackage[colorlinks,urlcolor=blue]{hyperref}
\usepackage{fullpage}

\title{Academic job market: how to maximize your chances}
\author{Irina Gaynanova}

\begin{document}

\maketitle

This document is based on my experience applying for a tenure-track Assistant Professor position in research university in the US during 2014-2015 job application cycle. The advice is specific to the field of Statistics, however some comments apply quite generally. All the expressed opinions are my own, and reflect my prospective as a candidate on the market.

{\color{red} \textbf{Update}: Since the original writing of this document, I had an opportunity to serve twice on the hiring committee in the Department of Statistics at Texas A\&M Univeristy, and observe faculty discussions during multiple hiring cycles. I have also observed that the market has changed, and 2021-2022 cycle has many differences from 2014-2015. I've heard that this document has been useful for many candidates, so rather than changing it from scratch, I am providing an update in each section in red to emphasize the difference in perspectives as a result of these experiences. All views are my personal opinions, and do not necessarily represent the general opinion of the hiring committee or the particular department.}

\section{Timeline}
This is a rough order in which things happen on the academic job market. The timeline is based on year 2014 and my own experience.
\begin{itemize}
\item Prepare core application materials (CV, teaching and research statements, website). Do this \textit{during summer}, so it's \textit{mostly ready by September}.
\item Decide on recommenders, need at least three (see specific advise below). Notify the recommenders in \textit{August/September}, and send them your application materials (at least CV and research statement)
\item Send applications (the first deadlines can be as early as \textit{November 1st}, most are \textit{middle of November} or \textit{December 1st}). Don't wait till the deadline since many places start looking earlier, aim for at least a week before the deadline. {\color{red} \textbf{Update}: there has been a tendency in many departments to start early to snatch up the best candidates. Many hiring committees start looking as early as October, so I will strongly recommend having your application submitted by October 15th for November deadlines, and possibly even earlier depending on what is in the ad.}
\item Prepare job talk. Have it ready \textit{by December}. {\color{red} \textbf{Update}: with a switch to early interviews in many places, I suggest having it ready by the middle of November to be on the safe side}
\item Phone interviews. Mostly in \textit{November and December}.
\item On-site interviews. Can  \textit{start in December} and \textit{go till March}. Most commonly happen \textit{throughout January and in early February}. {\color{red} \textbf{Update:} can start in November, and be busy in December already. But January is still the most popular month.}
\item Receiving offers. Early offers \textit{may happen in January}. Most \textit{start coming end of January}. {\color{red} Some places will start making offers in December to snatch good candidates early}
\item Making decisions. Usually you have two weeks since the offer has been made {\color{red}, but it's also common to have less time (10 days or even 7 days)}. Extensions are extremely hard to negotiate.
\end{itemize}

\section{Application Materials}

\subsection{CV}

Papers are the most important part of your CV and I believe they were the weakest part of my own application package. At the time, I only had papers in review and in progress, nothing was accepted yet. My recommendation is to divide the papers section in three parts: published/accepted (peer-reviewed journals are the ones that count), in review and in progress. For the papers that are in review, include the corresponding links to arXiv (and if you don't have them on arXiv, make sure to put them there). For the papers in progress, they really have to be in progress, i.e. you have a pretty good idea on how to approach the problem and when do you expect to be done and submit. Organize those papers by expected submission date.

{\color{red} Update: The market has changed dramatically. If you have no papers accepted in a statistics journal - you need to get a postdoc and get some papers out. There are increasingly more candidates applying with a postdoc (hence on average more papers on their CV), and even fresh PhD candidates tend to have multiple papers under their belt. Because of this change, I no longer recommend putting papers in progress on your CV as it may be perceived as CV padding. You should still put arXiv ones on your CV. People looking at the papers in your CV will automatically pay attention to the authors' order (first author counts the most), and the venue (Stat/ML vs domain-specific). If you have a paper where the order is alphabetical, indicate this on your CV. If you have a paper where you mentored another student, indicate this on your CV. If the paper is domain-specific, but you made a considerable contribution, it may still be worth indicating what the contribution was. These indications can be easily accomplished with one separate line after the paper item, e.g. in cursive, providing the explanation.}

There are two approaches to organizing your talks/presentations: by subject and just by date. In my case, I gave several presentations on the same subject at different places, and I hated to repeat the name of the presentation over and over again. On the other hand, I found that it's quite common to just give the dates, places and names of the conferences without specifying the subject, which is common in CVs at a more advanced academic stage. Choose whatever you like the best. If you gave some invited talks (i.e. invited by the department in a different university or invited section at a conference), make sure to separate those under Invited (with the rest being Contributed or Other). {\color{red} After looking at many CVs, I admit that I paid little attention to presentations. However, not having any stands out in a negative way (why not?). Having multiple invited ones stands out in a positive way.}

If you have hobbies, I will strongly recommend putting them in the end of your CV. If nothing else, it gives something to talk about for people who are not in your research area. In my case, I have discussed at least one of my hobbies with at least one of the faculty in each place I interviewed, even if we had similar research interests. I think it gives a better idea of your overall personality and makes it easier to connect with people.

{\color{red} \textbf{Formatting:} This is where flashy is not always good. A relatively boring CV with clear Education, Publications, Awards, Presentations, Teaching, Service sections is infinitely preferred to a CV that has strange fonts, wild colors, and overall unfamiliar structure. The latter are not very common, but they exist, and they tend to confuse more than help your case.}

{\color{red} \textbf{Website link:} Make sure to have a clickable link to your website on the first page of your CV. If you don't have a website, make sure you have one.  While a more complicated website may take time, you can have a simple but nice-looking website quite quickly thanks to Github, Google pages and multiple other places with free templates. Talk to your peers and advisors about the solutions they chose to find the one that works best for you. The website should have your photo, and some info about your research and teaching. I personally tend to check the websites of candidates I am interested in, and the impression is not the best if the one is lacking or outdated.}

\subsection{Research statement}

I have received drastically different advice on how long it should be and how much it should cover. I chose the combination that I liked the best and I believe my research statement was a strong part of my application package. This is the strategy I suggest.

Your research statement should be concise (3 pages at most), without formulas, complicated definitions or any other mathematical notation. A very broad range of people should be able to understand what you do. If you can't explain it well, than no one cares for your research. Your research statement should roughly have three parts: overview of your research interests, description of the research projects you have done and ideas for future work. These parts are aimed at answering three questions:  what kind of researcher you are and what problems interest you? (theoretical work, applied work, computing, something in between, application areas of particular interest), have you done solid research? (description of projects you have done), and most importantly are you ready to be on your own and where are you going with this? (in progress and future work). With the last part, it's important to demonstrate the balance between what you realistically know how to do (or have a very good idea on how to approach the problem successfully) and what you think will be super cool to do (i.e. missing in the current literature), but the strategy is less clear. In my statement, I had an overview (two paragraphs with the description of my research philosophy, interests and strengths), description of two large finished projects (two paragraphs each), brief list of projects in progress and for future work (one paragraph for each).

No matter what you write, make sure that it reads and sounds like you. If you like applied work, don't try to sell yourself as a theoretician. If you have strong interests in computing and found a nice interplay between computing and applicability of methods, just say that. Make sure that your research personality comes through. 

Finally, every word matters, so make it count. Avoid watery and imprecise statements. The best way to spot those is to give your research statement to someone who you know is critical but not harsh, and ask for an honest opinion on the strengths and weaknesses. If you are lucky like I was, it will make your statement much clearer and stronger.

{\color{red} Update: People read research statements. But only if your CV looks sufficiently good. Also, it's common to skim and read just the first paragraph for people who are not in your area, or look at figures (if any). So make sure the first paragraph is very good, and perhaps include a figure or two for visual appeal.}

\subsection{Teaching statement}

If you are applying to a research university, than the most common thing that people tell you is that nobody reads teaching statements. If you are applying to liberal arts school or a teaching position, then the roles of teaching statement and research statement are reversed. In any case, your statement should not be horrible, and writing your own is a great way to organize your thoughts on teaching. 

{\color{red} Update: As a member of the hiring committee and as the one who cares about teaching, I always read the teaching statements of the candidates I am interested in based on their CV. In my experience, a teaching statement is a place where unsupported claims, lack of care, and arrogance can often be spotted (and I largely read them because of this). But to be clear, I have also read many amazing teaching statements, which made me advocate for these candidates even more.} 

Composition wise, it should be no longer than 2 pages (3 pages if position is mostly teaching), and roughly answer the following questions about you: what do you like about teaching?, what is your approach in teaching statistics?, what are some common problems in teaching statistics and how would you face them?, what strategies do you like to use?, etc. The importance of particular questions depends on your teaching experience. Try to back up everything you say with particular examples and strategies you have used. The way I have started is by assembling a list of questions you may want to answer in a teaching statement (I believe University of Minnesota career website has good resources on this), made short answers to each of them and then picked the ones which I felt most strongly about. You don't need to try to put every single thought you have on teaching in one short document, but the points you put should be of a higher importance to you. I have repeatedly heard from a variety of people that the best ways to destroy your teaching statement is to make it boring and without context. Please don't use cliche statements as ``provide inclusive environment'' and ``incorporate technology in the classroom'', unless you have very concrete examples of how you did exactly that and what was the benefit to the students.

As with the research statement, in the end of the day make sure it sounds like you. My teaching statement ended up being much less formal than my research statement with some quite strong opinions, but after reading it through over and over again I knew that I had examples and experience with which I can back up every one of my points.

If you have no teaching experience and apply for a position that will require you to teach - I am not really sure what to recommend. Try to get any kind of experience (tutoring, community outreach), even grading can help. Some places will not even consider candidates with no prior teaching experience (which is justified in my opinion).

\subsection{Recommendation letters}

Majority of the places require 3 recommendation letters, whereas some allow you to submit up to 5. It is expected that one of these letters is from your PhD advisor, however you have a lot of freedom in whom you choose for the others. Some people by default choose their thesis committee members. I will not recommend automatically doing it until you carefully think about all the options you have.

First of all, think of faculty members who know you well. Are there professors that you taught or graded for? Are there professors with whom you took graduate level classes that required class project? Are there professors with whom you had some interesting discussions on research and teaching, even though they are not members of your committee? Any collaborative projects outside of your thesis? Don't limit yourself only to your department, think about faculty from other departments that you have interacted with. Have you participated in statistical consulting center? Your first task is to come up with the list of people whom you can potentially ask for recommendation in addition to committee members.

After you have this list of people, there are certain aspects that you want to consider.
\begin{itemize}
\item Which of your strengths will the person be able to comment on? Your teaching? Your thesis research? Your consulting skills? Your skills as a collaborator? Writing and general research skills? (i.e. if you had a class project) Ideally, your recommendations jointly should cover as much as possible.
\item How well known is the person in a statistics community? (assuming you are applying to Statistics Departments) Is it a junior researcher or a senior well-known professor? Generally, the recommendation of a senior person has stronger weight, however don't just go for the most senior person you know. See below.
\item Does this person have another student who is on the same academic job market this year? This is a very important point. If the answer is yes, how would you compare to the other student? If you work on very different things there should be no problem, but if the things are similar and the comparison is not in your favor, I will not ask that person for the recommendation.
 \end{itemize}

Contact people about references at least 3 months in advance, so that they have time to write a good recommendation. Different jobs may require different people. 


\subsection{Cover letter}
This is your introduction to the employer, which essentially summarizes why you are a good fit for the position and why do you want to join this department. I have heard that many places don't read them at all, however it doesn't mean you can write a bad one. Many positions emphasize a particular research area, and the cover letter gives a great opportunity to showcase that your research meets the position description. This is also a good place to draw attention to any significant awards you have (if any) and highlight your research. In case the place is attractive to you for a very specific reason (i.e. geographical location that makes you close to the family), it makes sense to mention it. If you are a strong candidate, some universities may not invite you thinking that you are unlikely to accept the offer in case it's made. Giving them additional reasons why you are interested in the position will maximize your chances of getting an interview.

If possible, have several sentences that can be personalized (i.e. a sentence on how you meet the position description and a sentence on why you like this particular department and university). The rest you can keep pretty much unchanged. Given how many applications you will likely submit and the low priority that is put on the cover letter, it doesn't make sense to spend too much time on this. Your time is better spent polishing your research statement and presentation slides.


\section{Where to look}
There are several places to look for the tenure-track Assistant professor positions. In Statistics, I found the following four sources to be the most useful.
\begin{itemize}
\item Job listing of the Institute of Mathematical Statistics. The ordering of the advertisements is not perfect, but most of the academic positions will be listed on this website.
\href{http://jobs.imstat.org/home/index.cfm?site_id=1847}{IMS job listing}

\item Job listing of the American Statistical Association. This listing has a lot of industry positions in addition to academic positions. \href{http://jobs.amstat.org/jobseekers/}{ASA job listing}

\item Job listing of the University of Florida Department of Statistics. This listing has a lot of postdoc positions in addition to academic positions. \href{http://www.stat.ufl.edu/jobs/}{UF Statistics job listing}

\item Job listing of academic jobs in Math. This listing has a lot of postdoc and academic positions primarily for the Math departments, which often are combined with Statistics and as such look for people with Statistics background.
\href{https://www.mathjobs.org/jobs}{MathJobs listing}
\end{itemize}
There are some additional sources, but they tend to have less jobs and overlap with the above listings. In my experience, the sources above covered all the listings that were of interest to me.

\section{Interviews}

\subsection{Phone/{\color{red} ZOOM} screening interviews}
{\color{red} These are screening interviews to make sure that you "in person" have a sufficiently good match to you "on the CV", and to gauge your interest in the position. The latter is particularly important as it's often harder to convey enthusiasm and personality remotely, so your best bet is to be prepared to answer obvious questions (see below), and be ready to ask questions of your own. As remote interaction goes both ways, these interviews may appear more dry and short, so I do not recommend you eliminate the place just because of the screening interview.}
Not every place has one, but it's becoming more and more common. I think I have 3 of those. Make sure you know who you are going to talk to and have ready answers for obvious questions like why are you interested in applying and why do you think you are a good fit. Make sure you can talk in a quiet place without interruption.

\subsection{On site interviews}
Usually last 1-2 full days, the main part of which is the job talk (see separate advice below). The other activities include individual meetings with faculty members, sometimes students and various lunches/dinners. {\color{red} Typically, you receive your schedule only a day or two before the visit, so you may want to start looking people up in advance to help you prepare. If you are interested in meeting specific faculty, or potential collaborators from other departments, make sure to convey this in advance as the departments will try to accommodate you. This may be particularly helpful in getting a feel for possible collaborators you may have, and establishing first contacts.}

All of my interviews went really well and I believe a large part of that success was my preparation. I had a notebook with a page dedicated to each person I was going to meet. On that page I put a short information about that person (title, PhD granting university, research interests, any significant administrative roles like graduate advisor). After that, I had a list of 3-4 questions I can ask that person. Some people you meet will not let you say a word, some people will look at you in anticipation of questions. I can not emphasize enough how useful that was in my interviews. Not only was I able to get a lot of information about each place, but I also avoided weird pauses when it was not clear what to talk about.

Here are some examples of the questions you may have:
\begin{itemize}
\item What do you like the best about the department?
\item If you had an opportunity to change something, what will it be?
\item Do you have collaborations within the department? Outside of the department?
\item How is the department socially?
\item How strong are the graduate students? How are they supported?
\item How many students are advised by a single professor?
\item Are undergraduate students involved in research?
\item Do you get graders/TAs?
\item What are the recreational activities?
\item What are the fun things to do around?
\item What is the amount of service you do?
\end{itemize}
The sky is the limit. There are some questions that you would like to reserve for the department head specifically:
\begin{itemize}
\item Tenure decision. Stat versus discipline journals. Grant expectations. Teaching expectations. You want to have a very precise idea of what is expected. {\color{red} I recommend you ask this question multiple times to get answers from different people (Dep Head, a senior professor who had experience serving on Promotion and Tenure committee, an Associate Professor who was recently tenured, an Assistant professor who is coming close to tenure) so you have the most accurate idea of the expectations and the process.}
\item Is there a mentoring program for new assistant professors? how does it work? {\color{red} Again, I recommend you ask Assistant Professors themselves in addition to Dep Head to get the most accurate representation.}
\item What is the plan for the department in the future? How does it fit the College?
\item When should I expect to hear from you?
\end{itemize}
There are also some questions that you would reserve for the dean
\begin{itemize}
\item Tenure decision. Be on the lookout for any discrepancies between what you hear from the dean and what you hear from the department head. It could happen and it's not a good sign.
\item How does the department fits the College? How does the College fits the University?
\item Are there internal funding opportunities?
\item What are the benefits? (health insurance, retirement plans, etc.)
\end{itemize}

Talk to faculty about their research and how you can help them to build on that. People are excited about people who can help them publish or who are fun to hang out with. People are trying to hire a colleague. Find out as much as you can about the people you are going to meet. Ask to meet graduate students, some newly hired faculty (potentially from other department), some people with similar research interests (if not already in the list).

This should be obvious, but be NICE to EVERYONE. This includes students, administrative assistants and anyone else you meet during the interview. This is a small world.

After an interview, send thank you emails. Either one to everyone, separate to everyone or one to the head of search committee asking to thank everyone. DON`T forget anyone.

\subsection{Job talk}

Practice makes it perfect, make sure it's ready by {\color{red} early } November (many places start doing interviews as early as December, {\color{red} and even end of November}). If you get any feedback during the interview, update the slides for the job talk accordingly.

I will recommend choosing one big topic that you can use to connect several of the projects you have done or one biggest project. Job talk is the most important part of your interview, so make sure you spend time on it, especially if presentations are not your strongest skill. First, start as simple as you can using as many real examples as you can. Try to explain things from first principles. People love to learn new things and if they can't follow your presentation after 5 minutes, they are not going to like you. At least the first 15 min of your talk should be accessible to anyone in the audience (including the students). You can start to build up more complicated things after that, but again, don't overcomplicate stuff. In the end, I had a slide that showcased some of my other projects. I believe it was beneficial to show people that there are other things you are interested in and can do.

Handling questions is also very important. Don't rush the answer and always make sure you heard the question before coming up with the answer. Also, don't try to be defensive. If you know the answer to the question, try to be short. If you disagree, try to soften your disagreement by saying something like "This is an interesting point, but we haven't explored it" or "Thank you for pointing this out, our approach was motivated by .. and so we haven't looked at ..". Be very mindful of time. If you feel like you are getting sunk into the argument, try to resume your presentation by saying something like "This is an interesting discussion, but in the interest of time let's continue after this talk". 

Usually you will have around 50 min for your presentation. It's always great to finish earlier and it's NEVER ok to take more time. People get tired and bored, you don't want to leave that impression.

{\color{red}
\subsection{What can go wrong}

In my experience, the most common reason for a bad interview is a terrible job talk, albeit other reasons are also possible (see below). I should also clarify that my "terrible" translates to "not good" for many people. These are some common pitfalls that can ruin your talk:

\begin{itemize}
\item Lack of background. It's unlikely that you will impress the audience by going straight into technical details and jargon. While faculty with similar research interests may still be able to follow your talk, you are losing a lot of other people in the audience. Furthermore, the talks are often used as a proxy measure of teaching ability, and it is not judged very high if half of the faculty have no idea what you are talking about. Depending on the area, the concepts that are basic to you may not be that basic to other people, so don't expect that everyone knows right away what you mean by propensity score matching, robust PCA , reinforcement learning, debiasing in sparse models, optimal transport, etc. You do not want to spend all your talk on definitions but having a couple background slides up-front to set the stage is incredibly helpful.

\item Lack of substance. To clarify, I do not mean a statement of a complicated-looking theorem with all assumptions and the proof. What I mean is that sometimes candidates decide to cover all of the projects they have ever done (please don't) without spending enough time talking about any of them in depth. You should be able to clearly convey what is important about the problem and why your solution is novel. This can happen to both theoretical and applied talks. In a theoretical talk, it is important to provide some practical motivation for the problem (importance) but also emphasize at a relatively high level the novelty of the solution. A theorem statement by itself does not accomplish this, and neither does a slide full of proof derivation. Make a contrast with existing results, why the result is challenging, and new insights that it provides. For an applied talk, it is important to demonstrate some methodological "meat" so you are not perceived as just applying others methods on various datasets. Talk about what challenges of methods 'transfer' and how you solved them, emphasize and highlight new methodological innovations, and convey overall impact of the work.

\item Poor question handling. It is expected that your job talk is your most well-prepared talk, so any freezing on questions or long wavy answers make you stand out from the other candidates who prepared better. If you understand the question, but you have no idea how to answer it, you shouldn't just wait for inspiration or try to come up with the answer on the go. If a limitation of your work is pointed out that you agree with, just acknowledge it (and elaborate if you plan to address it in future work). If a question is asked about how it compares to another work that you are not very familiar with, just state that you haven't thought about it but it will be interesting to compare and you would like to continue this discussion one on one. Of course, sometimes your reluctance to answer questions directly will be perceived against you (e.g. you can not recall a certain proof step or how long it takes to run the method on the data), but your best bet here is to know the work you present in and out.

\item Social attitude. This is judged not just from the talk, but the talk definitely adds. Here you should know yourself the best (and perhaps ask your friends if you are not sure) to be aware of which of the two pitfalls you are most likely to gravitate towards. The first pitfall is creating a perception of the lack of confidence. If you keep thanking your advisors throughout the whole talk, apologizing continuously for not knowing the answer, and avoiding giving any praise to your results, then the overall perception you create is that you are not ready to be on your own. If you are nervous speaking in public, your best bet here is to practice a lot, and continuously seek feedback from your advisors and friends.  The second pitfall is creating a perception of arrogance, and that your solution is the only solution. This typically occurs as a result of oversell, limited literature review, and disregard for any alternative solutions. Check whether faculty in the department that you are interviewing at have works directly related to your research, and make sure you acknowledge those works.
\end{itemize}

While the job talk is perhaps the most influential part of your interview (and the part that often changes pre-interview/post-interview candidates rankings), there are other aspects that can go wrong.

\begin{itemize}
\item Perceived lack of interest in the position. You will have plenty of opportunities to ask questions, and the lack of those is negatively perceived. Similarly, it is a negative sign if you have no idea who is on your schedule. In case any conversation leads to an opportunity for the follow up (e.g. sending an article on the topic, scheduling a meeting with potential collaborator), you better follow up if you are in fact interested in the position.



\item Social attitude. See the advice on talk. You want to leave the impression that you will be a nice colleague. Express interest in the department and faculty. Show some of your personality (e.g. ask if there is a climbing gym if that is one of your hobbies) so that others can see you as a future colleague.

\item Over-interest that is not backed up, and dragging through the negotiation process. This doesn't hurt your chances of getting an offer, but it does hurt your reputation, which is important long-term. I strongly recommend you avoid at all costs telling someone that they are your first choice during the interview if they are in fact not, and even if they are. The reason is that until you are presented with concrete offers and can evaluate all your options, you can't honestly assess which place you will prefer. If you really like the place, just say you really liked it and had very positive impressions (without explicitly providing a ranking).  Also, if you have two competing offers and you have a strong preference for one over the other, reject the other one to give the department an opportunity to hire another strong candidate, and to help other candidates on the market. I personally consider it highly unethical when the candidates' sit on multiple offers continuing to negotiate to the extreme the offer conditions and decision deadline when they are not seriously considering half of the places that the offers are coming from. To be clear, it's ok to take time to decide between a couple of offers when you are really not sure, and it's ok and expected that you will negotiate to help you make a decision and to ensure you have what you need. But you only should do it on the offers that you are seriously considering. Once you know which ones you will not take, reject without trying to negotiate or waiting for the expiration date. Again, the word is really small, and people remember.
\end{itemize}


Finally, sometimes you do everything right, but there is another candidate who also did everything right and is stronger, so you didn't get an offer. Do not despair. Interviews are also excellent networking opportunities, and again, the world is small. 

}

%\section{Negotiating offers}
%
%They always expect it. Things to negotiate: personal salary (ok to up 5-10\%, have a median numbers available, look up the university website, state school usually have this information open), start-up package. Whenever you ask for things/raise, be prepared to explain what you are going to use them for and why (even up to why you need a Mac), back it up. 
%
%Whenever you sign the offer, you are done with negotiation.
%
%
%\section{Making a choice and following up}



\end{document}